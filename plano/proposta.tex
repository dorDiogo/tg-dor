%        File: planoic.tex
%     Created: Fri Apr 18 01:00 PM 2014 B
% Last Change: Fri Apr 18 01:00 PM 2014 B
%
\documentclass[12pt, a4paper, oneside]{article}

\usepackage{amsfonts}
\usepackage[]{amsmath}
\usepackage{amssymb}
\usepackage[brazil]{babel}
\usepackage[left=20mm,right=20mm,top=25mm,bottom=25mm]{geometry}
\usepackage[]{graphicx}
\usepackage[utf8]{inputenc}
\usepackage{multirow}
\usepackage{paralist}
\usepackage{setspace}
\usepackage[compact]{titlesec}
\usepackage[]{url}
\usepackage{xspace}
\usepackage{mathpazo}
\usepackage{txfonts}

%\usepackage[]{biblatex}

\titleformat{\section}
  {\Large\bfseries}{\thesection.}{.5em}{}[\hrule\bigskip]
\titleformat{\subsection}
  {\bfseries}{\thesection.}{.5em}{}
%\renewcommand{\baselinestretch}{1.5} 
%\renewcommand{\rmdefault}{phv} % Arial
%\renewcommand{\sfdefault}{phv} % Arial
\setlength{\parskip}{6pt}

\newcommand{\A}{\ensuremath{\mathtt{A}}\xspace}
\newcommand{\C}{\ensuremath{\mathtt{C}}\xspace}
\newcommand{\G}{\ensuremath{\mathtt{G}}\xspace}
\newcommand{\T}{\ensuremath{\mathtt{T}}\xspace}

\newcommand{\str}[1]{\ensuremath{\mathtt{#1}}\xspace}
\newcommand{\strset}[1]{\ensuremath{\mathcal{#1}}\xspace}
\newcommand{\ssS}{\strset{S}}
\newcommand{\seq}[1]{\ensuremath{\mathtt{#1}}\xspace}

\newcommand{\rank}{\ensuremath{\mathrm{rank}}\xspace}
\newcommand{\select}{\ensuremath{\mathrm{select}}\xspace}
\newcommand{\BWT}{\ensuremath{\mathrm{BWT}}\xspace}

\newcommand{\X}{\ensuremath{\medbullet}\xspace}
\newcommand{\x}{\ensuremath{\medcirc}\xspace}

\newcommand{\pending}[1]{\textbf{PENDING:\[#1\]}}


%\bibliography{projeto}


\begin{document}
%\onehalfspacing

%\maketitle

\thispagestyle{empty}
\begin{center}
\Large
Universidade Federal de Pernambuco\\
Centro de Informática\\
Graduação em Engenharia da Computação

\vfill

{\huge \bfseries Busca aproximada de texto baseada em índices de minimisers}
\\
\medskip
{\bfseries\itshape Proposta de Trabalho de Graduação}

\vfill

\bigskip

	\begin{tabular}{r p{100mm}}
	\textbf{Aluno: } & Diogo Oliveira Rodrigues \newline(\texttt{alpgc@cin.ufpe.br}) \\ 
\textbf{Orientador: } & Paulo Gustavo Soares da Fonseca \newline(\texttt{paguso@cin.ufpe.br})
\\
	\textbf{Área: } & Teoria da Computação / Algoritmos
\end{tabular}

	\vspace{3cm}
Recife, Agosto de 2019
\end{center}

\clearpage 
\thispagestyle{empty}
\section{Resumo}

Um problema algorítmico básico consiste em procurar ocorrências aproximadas de um \emph{padrão} $P$ num \emph{texto} $T$ (ou coleção de textos) de muito maior tamanho. Quando o texto é fixo, um \emph{índice} $I(T)$ é construído de forma a permitir que ocorrências de padrões sejam nele encontradas sem a necessidade de se percorrê-lo por completo. 
Diversos índices foram propostos na literatura, com relações diversas entre espaço de memória e tempo de busca. Em aplicações com grandes volumes de dados, como em Biologia Computacional, índices completos tradicionais como árvores de sufixos ou vetores de sufixos, que representam todas as subcadeias do texto, podem requerer um espaço proibitivo. Alternativamente, alguns índices parciais recentemente propostos representam apenas um subconjunto de subcadeias de tamanho fixo $k$, chamados \emph{minimisers}. Esses índices são combinados a heurísticas de busca que permitem localizar ocorrências aproximadas de um padrão a partir de extensões de ocorrências exatas de subcadeias de tamanho fixo. Em geral, temos como características conflitantes o tamanho do índice, sua relação entre sensibilidade e especificidade, e o tempo da busca. Esses fatores são diretamente influenciados pela escolha de quais, quantos e de que tamanho serão os minimisers indexados, e de como serão combinados para a busca. O objetivo deste trabalho é estudar a relação entre esses fatores e propor algoritmos e estruturas de dados que permitam uma parametrização flexível para uma busca eficiente e com espaço reduzido. 



\clearpage
\setcounter{page}{1}
\section{Introdução}

Um problema básico de processamento de texto, com aplicação em diversas áreas, consiste em procurar ocorrências aproximadas de uma cadeia, ou \emph{padrão}, $P$ num texto $T$ (ou coleção de textos $\mathcal{T}$) de muito maior tamanho. Por exemplo, em Biologia Computacional, costuma-se alinhar um determinado fragmento de sequência de DNA a um genoma ou conjunto de genomas de referência para identificar regiões homólogas em outras espécies. 

Quando o texto é fixo, um \emph{índice} $I(T)$ é construído de forma a permitir que ocorrências de padrões sejam nele encontradas sem a necessidade de se percorrê-lo por completo. Um dos índices mais simples consiste numa tabela contendo ponteiros para as ocorrências no texto de subsequências de comprimento fixo $k$, ou \emph{kmers}. Para procurar $P$ em $T$, primeiro consultamos a tabela para descobrir as localizações no texto de um kmer presente no padrão, chamado \emph{semente} (\emph{seed}), e logo confirmamos a ocorrência de $P$ em cada uma dessas posições, estendendo o alinhamento original da semente pelos flancos. A essa estratégia dá-se o nome de \emph{seed and extend}. 

Existe um compromisso entre o tamanho $k$ dos kmers indexados e o tamanho do índice, já que o número de diferentes kmers aumenta exponencialmente com $k$. Por outro lado, existe também uma relação entre $k$ e a sensibilidade e especificidade do índice. Um $k$ pequeno implica numa menor quantidade de kmers indexados, mas cada um deles terá uma maior quantidade de ocorrências, o que poderá tornar o índice muito sensível porém pouco específico, com repercussão negativa sobre o tempo da busca. Já um valor demasiado grande para $k$ implica em menos kmers indexados, mas o índice pode tornar-se pouco sensível e muito específico, levando à não-detecção de ocorrências aproximadas significativas. Escolher o valor adequado para $k$ de forma a equilibrar o tamanho do índice, sua relação entre sensibilidade e especificidade, além do tempo da busca, constitui-se portanto num importante problema.

\subsection*{Minimisers}

Considere um texto $T=t_0\cdots t_{n-1}$ de comprimento $n$ e denotemos por $T[i:j]=t_i\cdots t_{j-1}$ a subcadeia de comprimento $j-i$\/ iniciada na posição $i$. Uma tal subcadeia de comprimento $w+k-1$ corresponde a uma janela de $w$\/ kmers consecutivos. Considerando uma relação de ordem qualquer sobre o conjunto dos kmers (por exemplo, a ordem lexicográfica), chamamos o kmer mínimo dessa janela do seu $(w,k)$-\textit{minimiser}, ou simplesmente minimiser quando os valores dos parâmetros $w$ e $k$ estiverem subentendidos. A Figura~\ref{fig:minimiser} ilustra esse conceito.

\begin{figure}[htpb]
	\centering
	%\includegraphics[width=0.8\linewidth]{name.ext}

	\begin{verbatim}
                                Janela
                           |<-- (w+k-1) -->|  
        Texto T: ... ACGTAGCATGATCGATCGATCGAGCTTAGCTAGCTAGCAT ...
                               *   *	
kmers na janela:           CATGATC
                            ATGATCG
                             TGATCGA
                              GATCGAT
                               ATCGATC  <------+-- (11,7)-minimiser 
                                TCGATCG        |
                                 CGATCGA       |
                                  GATCGAT      |
                                   ATCGATC  <--+
                                    TCGATCG
                                     CGATCGA
	\end{verbatim}
	\caption{Exemplo de minimiser para $w=11$ e $k=7$ baseado na ordem lexicográfica. As posições onde o minimiser ocorre no texto são indicadas por $*$. Repare que o minimiser pode ocorrer mais de uma vez na janela, inclusive com sobreposição.}
	\label{fig:minimiser}
\end{figure}

Schleimer et al \cite{Schleimer2003} e, independentemente, Roberts et al \cite{Roberts2004} propuseram a utilização dos minimisers para construção de índices de tamanho reduzido. A ideia básica é a seguinte. Primeiro, desliza-se uma janela de $w$ kmers consecutivos (ou seja, uma janela de comprimento $w+k-1$) por $T$ e, para cada uma delas, armazenamos o seu minimiser junto com as suas ocorrências numa tabela. Posteriormente, para procurar por $P$, desliza-se igualmente uma janela de mesmo tamanho sobre o padrão e, em cada posição, utiliza-se o minimiser da janela como semente. Com efeito, se $P$ e $T$ compartilharem uma mesma subsequência de comprimento $\geq w+k-1$, então o minimiser dessa subsequência deverá ser encontrado na tabela. Em outras palavras, toda ocorrência do padrão com alinhamento exato de uma subsequência de comprimento mínimo $w+k-1$ será identificada. Portanto, esse valor controla a relação entre sensibilidade e especificidade do índice.

Nos trabalhos seminais \cite{Schleimer2003, Roberts2004} também foi demonstrado que, sob hipótese de equiprobabilidade e independência de escolha de cada kmer da janela como sendo o minimiser, na média um novo minimiser é escolhido a cada $(w+1)/2$ deslocamentos sucessivos da janela. Assim, o parâmetro $w$ controla a taxa de amostragem dos kmers indexados e logo influencia o tamanho do índice. Segue-se, como referido acima, que a eficiência e eficácia do índice dependem da criteriosa escolha dos valores de $w$ e $k$.

Ocorre que, em algumas situações, a escolha ótima dos parâmetros $w$ e $k$ pode ser muito desafiadora, senão praticamente impossível. Em aplicações de Biologia Computacional, por exemplo, muitas vezes as sequências a serem mapeadas (os padrões) apresentam características bastante variáveis quanto ao comprimento, composição e taxa de erro. Da mesma forma, algumas regiões do genoma têm maior ou menor complexidade, como regiões com longos homopolímeros (sequências de uma mesma letra), ou  regiões enriquecidas de pequenas repetições. Dependendo da combinação das características dos padrões e/ou regiões do texto, pode ser necessário utilizar valores diferentes para $w$ e $k$. Na prática, isso muitas vezes implica em rodar uma ferramenta de mapeamento diversas vezes, com diferentes parametrizações.



\clearpage
\section{Objetivos}

\pending{\cite{Li2016}\cite{Fernandes2011}}



As estruturas e algoritmos propostos serão implementados e o  software desenvolvido no projeto será acrescentado a uma biblioteca básica em escrita em C, atualmente em desenvolvimento, para que possa ser útil à comunidade de processamento de texto e Biologia Computacional.


\clearpage
\section{Metodologia}

O projeto será organizado nas atividades descritas abaixo e desenvolvidas conforme o cronograma a seguir.

\subsection*{T0. Preparação da proposta}

Nesta fase inicial, aluno e orientador farão uma série de reuniões para definição do problema e escopo do projeto. O orientador apresentará a bibliografia básica sobre o tema e os possíveis pontos a serem abordados, e ambos decidirão sobre aqueles a serem desenvolvidos com base no  interesse mútuo.

\subsection*{T1. Revisão e acompanhamento bibliográfico}

Essa tarefa será executada de maneira mais acentuada no início do projeto. A atividade consiste num estudo dirigido centrado no  material bibliográfico mais diretamente relacionado às estruturas de dados e algoritmos a serem implementados no projeto. Ao final dessa fase inicial, espera-se que o aluno possa manter-se atualizado e aprofundar-se em pontos específicos de maneira mais autônoma.


\subsection*{T2. Implementação dos algoritmo e estruturas}

Esta tarefa corresponde à principal componente de desenvolvimento do projeto. O aluno deverá, em interação com o orientador, estudar detalhadamente algoritmos e estruturas de dados relativos aos índices e grafos de subsequências, e desenvolver uma implementação de referência em nível de produção para os mesmos, com base em uma biblioteca em C atualmente em desenvolvimento.


\subsection*{T3. Realização de testes e experimentos}

Nesta tarefa, o aluno deverá fazer uma análise experimental comparativa do desempenho dos métodos propostos e implementados relativamente a outras alternativas públicas identificadas num levantamento inicial na tarefa T1. Pode ser necessário obter dados reais e produzir dados sintéticos que simulem uma situação limite de estresse para os métodos.


\subsection*{T4. Redação e revisão da monografia}

A monografia produzida deverá conter:
(1)~uma breve revisão bibliográfica do estado da arte, fruto da tarefa T1, (2)~descrição detalhada dos seus desenvolvimentos, incluindo a análise teórica dos algoritmos e estruturas propostas em termos de tempo e espaço, (3)~uma análise crítica com base em resultados experimentais da tarefa T3, e (4)~uma discussão com conclusões gerais sobre o projeto.


\subsection*{T5. Preparação da Apresentação}

Finalmente, o aluno preparará a apresentação da defesa do TG com o resumo dos desenvolvimentos e resultados obtidos.

\medskip

O aluno já possui alguma familiaridade com a área, tendo inclusive cursado uma disciplina eletiva oferecida pelo orientador, pelo que poderá por-se rapidamente em desenvolvimento. O acompanhamento será feito pessoalmente através de reuniões semanais. Todo material desenvolvido é compartilhado entre orientador e aluno através de um repositório privado na plataforma GitHub. 

\clearpage
\section{Cronograma de atividades}


\begin{center}
	\begin{tabular}{| l || c | c | c | c | c | c | c | c | c | c | c | c | c | c | c | c |  c | c | c | c | }
		\hline
		& \multicolumn{3}{| c |}{Ago} & \multicolumn{4}{| c |}{Set} & \multicolumn{4}{| c |}{Out} & \multicolumn{4}{| c |}{Nov} & \multicolumn{2}{| c |}{Dez} \\\hline\hline
		Preparação da proposta & \X & \X & \X & & & & & & & & & & & & & & \\\hline 
		Revisão bibliográfica$^\dagger$ & \X & \X & \X & \x & \x & \x & \x & \x & & & & &  & & & & \\\hline 
		Implementação das estruturas & & & &  \X & \X & \X & \X & \X & \X & \X & \X & \X & & & & & \\\hline 
		Realização dos testes$^\ddagger$ & & & & \x & \x & \x & \x & \x & \x  & \X & \X & \X & \X & & & & \\\hline 
		Redação e revisão da monografia & & & & & & & & & & & \X & \X & \X & \X & \X & & \\\hline 
		Preparação da apresentação & & & & & & & & & & & & & & &  & \X & \X \\\hline 
\hline
	\end{tabular}
\begin{minipage}{0.9\linewidth}
\noindent($\dagger$) \X = levantamento inicial, \x= aprofundamento\newline
\noindent($\ddagger$) \X= experimentos finais , \x = testes preliminares para orientar desenvolvimento\newline
\end{minipage}

\end{center}


\clearpage
%\nocite{*}
\bibliographystyle{unsrt-etal}
\bibliography{proposta}
%\printbibliography

\clearpage
\section{Possíveis Avaliadores}

\begin{enumerate}
\item Prof. Nivan Ferreira Jr 
\item Prof. Gustavo Carvalho 
\end{enumerate}


\clearpage
\section{Assinaturas}

\vfill
\begin{center}
	Recife, 22 de Março de 2019

	\vspace{3cm}
	\rule{10cm}{.5pt}\\
	\textbf{Aluno:} Arthur Latache Pimentel Gesteira Costa\\

	\vspace{3cm}
	\rule{10cm}{.5pt}\\
	\textbf{Orientador:} Paulo Gustavo Soares da Fonseca\\
\end{center}
\vfill

\end{document}

